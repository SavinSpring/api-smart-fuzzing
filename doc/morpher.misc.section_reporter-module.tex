%
% API Documentation for Morpher
% Module morpher.misc.section_reporter
%
% Generated by epydoc 3.0.1
% [Thu Nov 17 01:40:49 2011]
%

%%%%%%%%%%%%%%%%%%%%%%%%%%%%%%%%%%%%%%%%%%%%%%%%%%%%%%%%%%%%%%%%%%%%%%%%%%%
%%                          Module Description                           %%
%%%%%%%%%%%%%%%%%%%%%%%%%%%%%%%%%%%%%%%%%%%%%%%%%%%%%%%%%%%%%%%%%%%%%%%%%%%

    \index{morpher \textit{(package)}!morpher.misc \textit{(package)}!morpher.misc.section\_reporter \textit{(module)}|(}
\section{Module morpher.misc.section\_reporter}

    \label{morpher:misc:section_reporter}
\textbf{Author:} Rob Waaser



\textbf{Contact:} robwaaser@gmail.com



\textbf{Organization:} Carnegie Mellon University



\textbf{Since:} November 2, 2011




%%%%%%%%%%%%%%%%%%%%%%%%%%%%%%%%%%%%%%%%%%%%%%%%%%%%%%%%%%%%%%%%%%%%%%%%%%%
%%                               Variables                               %%
%%%%%%%%%%%%%%%%%%%%%%%%%%%%%%%%%%%%%%%%%%%%%%%%%%%%%%%%%%%%%%%%%%%%%%%%%%%

  \subsection{Variables}

    \vspace{-1cm}
\hspace{\varindent}\begin{longtable}{|p{\varnamewidth}|p{\vardescrwidth}|l}
\cline{1-2}
\cline{1-2} \centering \textbf{Name} & \centering \textbf{Description}& \\
\cline{1-2}
\endhead\cline{1-2}\multicolumn{3}{r}{\small\textit{continued on next page}}\\\endfoot\cline{1-2}
\endlastfoot\raggedright \_\-\_\-p\-a\-c\-k\-a\-g\-e\-\_\-\_\- & \raggedright \textbf{Value:} 
{\tt \texttt{'}\texttt{morpher.misc}\texttt{'}}&\\
\cline{1-2}
\end{longtable}


%%%%%%%%%%%%%%%%%%%%%%%%%%%%%%%%%%%%%%%%%%%%%%%%%%%%%%%%%%%%%%%%%%%%%%%%%%%
%%                           Class Description                           %%
%%%%%%%%%%%%%%%%%%%%%%%%%%%%%%%%%%%%%%%%%%%%%%%%%%%%%%%%%%%%%%%%%%%%%%%%%%%

    \index{morpher \textit{(package)}!morpher.misc \textit{(package)}!morpher.misc.section\_reporter \textit{(module)}!morpher.misc.section\_reporter.SectionReporter \textit{(class)}|(}
\subsection{Class SectionReporter}

    \label{morpher:misc:section_reporter:SectionReporter}
\begin{tabular}{cccccc}
% Line for object, linespec=[False]
\multicolumn{2}{r}{\settowidth{\BCL}{object}\multirow{2}{\BCL}{object}}
&&
  \\\cline{3-3}
  &&\multicolumn{1}{c|}{}
&&
  \\
&&\multicolumn{2}{l}{\textbf{morpher.misc.section\_reporter.SectionReporter}}
\end{tabular}

Extends \texttt{StatusReporter} with additional functionality for 
multi-part status bars

\texttt{StatusReporter} requires that you know the number of events that 
will be tracked at the time the status bar is created. However in some 
cases this information is not completely known. For example, a program 
might be written to process ten batches of files, but the number of files 
in the each batch is not known until the previous batch is completed. 
SectionReporter allows the status bar to be divided into a known number of 
sections, but the number of events tracked in each section does not need to
be known until that section is reached by the status bar. This allows the 
status bar to display quasi-accurate completion information and remaining 
time estimates even if the actual information is impossible to determine at
that time.

SectionReporter objects can also be reused multiple times by using the 
\texttt{start} method, which essentially resets the counter. The usage 
pattern is:

\begin{alltt}
   rep = SectionReporter(2)
   rep.start()
   rep.startSection(1, 10)
   ...call rep.pulse() ten times....
   rep.endSection()
   rep.startSection(2, 3)
   ...call rep.pulse() three times
   rep.endSection()\end{alltt}

\textbf{Warning:} The status bar assumes that no other output is sent to the console in 
dynamic update mode and will not display correctly otherwise



\textbf{See Also:} \texttt{StatusReporter} is the base class for this class




%%%%%%%%%%%%%%%%%%%%%%%%%%%%%%%%%%%%%%%%%%%%%%%%%%%%%%%%%%%%%%%%%%%%%%%%%%%
%%                                Methods                                %%
%%%%%%%%%%%%%%%%%%%%%%%%%%%%%%%%%%%%%%%%%%%%%%%%%%%%%%%%%%%%%%%%%%%%%%%%%%%

  \subsubsection{Methods}

    \vspace{0.5ex}

\hspace{.8\funcindent}\begin{boxedminipage}{\funcwidth}

    \raggedright \textbf{\_\_init\_\_}(\textit{self}, \textit{numsections})

    \vspace{-1.5ex}

    \rule{\textwidth}{0.5\fboxrule}
\setlength{\parskip}{2ex}
    Initializes a new object wrapping an underlying \texttt{StatusReporter}
    object using default settings

\setlength{\parskip}{1ex}
      \textbf{Parameters}
      \vspace{-1ex}

      \begin{quote}
        \begin{Ventry}{xxxxxxxxxxx}

          \item[numsections]

          The total number of sections tracked by the status bar

            {\it (type=integer)}

        \end{Ventry}

      \end{quote}

      Overrides: object.\_\_init\_\_

    \end{boxedminipage}

    \label{morpher:misc:section_reporter:SectionReporter:start}
    \index{morpher \textit{(package)}!morpher.misc \textit{(package)}!morpher.misc.section\_reporter \textit{(module)}!morpher.misc.section\_reporter.SectionReporter \textit{(class)}!morpher.misc.section\_reporter.SectionReporter.start \textit{(method)}}

    \vspace{0.5ex}

\hspace{.8\funcindent}\begin{boxedminipage}{\funcwidth}

    \raggedright \textbf{start}(\textit{self}, \textit{msg}={\tt \texttt{'}\texttt{  Status:}\texttt{'}})

    \vspace{-1.5ex}

    \rule{\textwidth}{0.5\fboxrule}
\setlength{\parskip}{2ex}
    Resets the internal counters, prints a message, and prints the empty 
    status bar.

\setlength{\parskip}{1ex}
      \textbf{Parameters}
      \vspace{-1ex}

      \begin{quote}
        \begin{Ventry}{xxx}

          \item[msg]

          The message to print just above the status bar, default is 
          "Status:"

            {\it (type=string)}

        \end{Ventry}

      \end{quote}

\textbf{Note:} The elapsed time is calculated from the last time this method was called 
for this object



    \end{boxedminipage}

    \label{morpher:misc:section_reporter:SectionReporter:startSection}
    \index{morpher \textit{(package)}!morpher.misc \textit{(package)}!morpher.misc.section\_reporter \textit{(module)}!morpher.misc.section\_reporter.SectionReporter \textit{(class)}!morpher.misc.section\_reporter.SectionReporter.startSection \textit{(method)}}

    \vspace{0.5ex}

\hspace{.8\funcindent}\begin{boxedminipage}{\funcwidth}

    \raggedright \textbf{startSection}(\textit{self}, \textit{section}, \textit{numevents})

    \vspace{-1.5ex}

    \rule{\textwidth}{0.5\fboxrule}
\setlength{\parskip}{2ex}
    Sets the current section to the given section number and sets the total
    number of events tracked by this section

    The variable "curevents" is dynamically scaled at this time as if all 
    previous sections had also tracked the same number of events

\setlength{\parskip}{1ex}
      \textbf{Parameters}
      \vspace{-1ex}

      \begin{quote}
        \begin{Ventry}{xxxxxxxxx}

          \item[section]

          The index of the section to start, beginning from 1.

            {\it (type=integer)}

          \item[numevents]

          Total number of events tracked by this section.

            {\it (type=integer)}

        \end{Ventry}

      \end{quote}

    \end{boxedminipage}

    \label{morpher:misc:section_reporter:SectionReporter:pulse}
    \index{morpher \textit{(package)}!morpher.misc \textit{(package)}!morpher.misc.section\_reporter \textit{(module)}!morpher.misc.section\_reporter.SectionReporter \textit{(class)}!morpher.misc.section\_reporter.SectionReporter.pulse \textit{(method)}}

    \vspace{0.5ex}

\hspace{.8\funcindent}\begin{boxedminipage}{\funcwidth}

    \raggedright \textbf{pulse}(\textit{self}, \textit{events}={\tt 1})

    \vspace{-1.5ex}

    \rule{\textwidth}{0.5\fboxrule}
\setlength{\parskip}{2ex}
    Increments the number of events completed by the given amount, or 1 by 
    default, then reprints the status bar.

\setlength{\parskip}{1ex}
      \textbf{Parameters}
      \vspace{-1ex}

      \begin{quote}
        \begin{Ventry}{xxxxxx}

          \item[events]

          The number of events to increment the counter by, default is 1

            {\it (type=integer)}

        \end{Ventry}

      \end{quote}

\textbf{Note:} The status bar will not actually reflect this section as being 100 percent 
complete until \texttt{endSection} is called.



    \end{boxedminipage}

    \label{morpher:misc:section_reporter:SectionReporter:endSection}
    \index{morpher \textit{(package)}!morpher.misc \textit{(package)}!morpher.misc.section\_reporter \textit{(module)}!morpher.misc.section\_reporter.SectionReporter \textit{(class)}!morpher.misc.section\_reporter.SectionReporter.endSection \textit{(method)}}

    \vspace{0.5ex}

\hspace{.8\funcindent}\begin{boxedminipage}{\funcwidth}

    \raggedright \textbf{endSection}(\textit{self})

    \vspace{-1.5ex}

    \rule{\textwidth}{0.5\fboxrule}
\setlength{\parskip}{2ex}
    Ends the current section, correcting the status bar to reflect exactly 
    \textit{(cursection/numsections)*100} percent completion

\setlength{\parskip}{1ex}
    \end{boxedminipage}


\large{\textbf{\textit{Inherited from object}}}

\begin{quote}
\_\_delattr\_\_(), \_\_format\_\_(), \_\_getattribute\_\_(), \_\_hash\_\_(), \_\_new\_\_(), \_\_reduce\_\_(), \_\_reduce\_ex\_\_(), \_\_repr\_\_(), \_\_setattr\_\_(), \_\_sizeof\_\_(), \_\_str\_\_(), \_\_subclasshook\_\_()
\end{quote}

%%%%%%%%%%%%%%%%%%%%%%%%%%%%%%%%%%%%%%%%%%%%%%%%%%%%%%%%%%%%%%%%%%%%%%%%%%%
%%                              Properties                               %%
%%%%%%%%%%%%%%%%%%%%%%%%%%%%%%%%%%%%%%%%%%%%%%%%%%%%%%%%%%%%%%%%%%%%%%%%%%%

  \subsubsection{Properties}

    \vspace{-1cm}
\hspace{\varindent}\begin{longtable}{|p{\varnamewidth}|p{\vardescrwidth}|l}
\cline{1-2}
\cline{1-2} \centering \textbf{Name} & \centering \textbf{Description}& \\
\cline{1-2}
\endhead\cline{1-2}\multicolumn{3}{r}{\small\textit{continued on next page}}\\\endfoot\cline{1-2}
\endlastfoot\multicolumn{2}{|l|}{\textit{Inherited from object}}\\
\multicolumn{2}{|p{\varwidth}|}{\raggedright \_\_class\_\_}\\
\cline{1-2}
\end{longtable}


%%%%%%%%%%%%%%%%%%%%%%%%%%%%%%%%%%%%%%%%%%%%%%%%%%%%%%%%%%%%%%%%%%%%%%%%%%%
%%                          Instance Variables                           %%
%%%%%%%%%%%%%%%%%%%%%%%%%%%%%%%%%%%%%%%%%%%%%%%%%%%%%%%%%%%%%%%%%%%%%%%%%%%

  \subsubsection{Instance Variables}

    \vspace{-1cm}
\hspace{\varindent}\begin{longtable}{|p{\varnamewidth}|p{\vardescrwidth}|l}
\cline{1-2}
\cline{1-2} \centering \textbf{Name} & \centering \textbf{Description}& \\
\cline{1-2}
\endhead\cline{1-2}\multicolumn{3}{r}{\small\textit{continued on next page}}\\\endfoot\cline{1-2}
\endlastfoot\raggedright c\-u\-r\-e\-v\-e\-n\-t\-s\- & The total number of events that have completed, across all 
          sections - dynamically scaled when a new section is entered as if
          all previous sections were composed of the same number of total 
          events&\\
\cline{1-2}
\raggedright c\-u\-r\-s\-e\-c\-t\-i\-o\-n\- & The current section index, starting from 1&\\
\cline{1-2}
\raggedright c\-u\-r\-t\-o\-t\-a\-l\- & The total number of events tracked by the current section&\\
\cline{1-2}
\raggedright n\-u\-m\-s\-e\-c\-t\-i\-o\-n\-s\- & The total number of sections making up the status bar&\\
\cline{1-2}
\raggedright r\-e\-p\-o\-r\-t\-e\-r\- & The encapsulated \texttt{StatusReporter} object used to print the
          status bar itself.&\\
\cline{1-2}
\end{longtable}

    \index{morpher \textit{(package)}!morpher.misc \textit{(package)}!morpher.misc.section\_reporter \textit{(module)}!morpher.misc.section\_reporter.SectionReporter \textit{(class)}|)}
    \index{morpher \textit{(package)}!morpher.misc \textit{(package)}!morpher.misc.section\_reporter \textit{(module)}|)}
