%
% API Documentation for Morpher
% Module morpher.misc.config
%
% Generated by epydoc 3.0.1
% [Thu Nov 17 01:40:49 2011]
%

%%%%%%%%%%%%%%%%%%%%%%%%%%%%%%%%%%%%%%%%%%%%%%%%%%%%%%%%%%%%%%%%%%%%%%%%%%%
%%                          Module Description                           %%
%%%%%%%%%%%%%%%%%%%%%%%%%%%%%%%%%%%%%%%%%%%%%%%%%%%%%%%%%%%%%%%%%%%%%%%%%%%

    \index{morpher \textit{(package)}!morpher.misc \textit{(package)}!morpher.misc.config \textit{(module)}|(}
\section{Module morpher.misc.config}

    \label{morpher:misc:config}
\textbf{Author:} Rob Waaser



\textbf{Contact:} robwaaser@gmail.com



\textbf{Organization:} Carnegie Mellon University



\textbf{Since:} October 22, 2011




%%%%%%%%%%%%%%%%%%%%%%%%%%%%%%%%%%%%%%%%%%%%%%%%%%%%%%%%%%%%%%%%%%%%%%%%%%%
%%                               Variables                               %%
%%%%%%%%%%%%%%%%%%%%%%%%%%%%%%%%%%%%%%%%%%%%%%%%%%%%%%%%%%%%%%%%%%%%%%%%%%%

  \subsection{Variables}

    \vspace{-1cm}
\hspace{\varindent}\begin{longtable}{|p{\varnamewidth}|p{\vardescrwidth}|l}
\cline{1-2}
\cline{1-2} \centering \textbf{Name} & \centering \textbf{Description}& \\
\cline{1-2}
\endhead\cline{1-2}\multicolumn{3}{r}{\small\textit{continued on next page}}\\\endfoot\cline{1-2}
\endlastfoot\raggedright \_\-\_\-p\-a\-c\-k\-a\-g\-e\-\_\-\_\- & \raggedright \textbf{Value:} 
{\tt \texttt{'}\texttt{morpher.misc}\texttt{'}}&\\
\cline{1-2}
\end{longtable}


%%%%%%%%%%%%%%%%%%%%%%%%%%%%%%%%%%%%%%%%%%%%%%%%%%%%%%%%%%%%%%%%%%%%%%%%%%%
%%                           Class Description                           %%
%%%%%%%%%%%%%%%%%%%%%%%%%%%%%%%%%%%%%%%%%%%%%%%%%%%%%%%%%%%%%%%%%%%%%%%%%%%

    \index{morpher \textit{(package)}!morpher.misc \textit{(package)}!morpher.misc.config \textit{(module)}!morpher.misc.config.Config \textit{(class)}|(}
\subsection{Class Config}

    \label{morpher:misc:config:Config}
\begin{tabular}{cccccccc}
% Line for ConfigParser.RawConfigParser, linespec=[False, False]
\multicolumn{2}{r}{\settowidth{\BCL}{ConfigParser.RawConfigParser}\multirow{2}{\BCL}{ConfigParser.RawConfigParser}}
&&
&&
  \\\cline{3-3}
  &&\multicolumn{1}{c|}{}
&&
&&
  \\
% Line for ConfigParser.ConfigParser, linespec=[False]
\multicolumn{4}{r}{\settowidth{\BCL}{ConfigParser.ConfigParser}\multirow{2}{\BCL}{ConfigParser.ConfigParser}}
&&
  \\\cline{5-5}
  &&&&\multicolumn{1}{c|}{}
&&
  \\
&&&&\multicolumn{2}{l}{\textbf{morpher.misc.config.Config}}
\end{tabular}

A wrapper for Python's \texttt{ConfigParser} class which adds some 
project-specific configuration and a toString method.

Inherits from Pythons' standard \texttt{ConfigParser} class, which is used 
to read in files in the well-known INI format and parse them for 
configuration information. Config overrides the \texttt{\_\_init\_\_} 
method with it's own version, which does some project-specific 
configuration, and also adds a toString method, which returns a 
pretty-printed string useful for logging the state of this Config object.

Config is designed to be used as a central registry of configuration 
information for a project, and after it is initialized with the contents of
a configuration file, it should be passed to every object in the project 
that needs to access configuration information. Each object can then use 
their individual reference to the global Config object to read and write 
key-value pairs as necessary, which can be seen by all other objects as 
well.

\textbf{Note:} Config is naturally pickleable as long as no key-value pairs are added that
contain pickleable objects - meaning it can be used to store a programs' 
state to a file and used to later restore that state.



\textbf{To Do:} Add additional validation of parameters read from the config file




%%%%%%%%%%%%%%%%%%%%%%%%%%%%%%%%%%%%%%%%%%%%%%%%%%%%%%%%%%%%%%%%%%%%%%%%%%%
%%                                Methods                                %%
%%%%%%%%%%%%%%%%%%%%%%%%%%%%%%%%%%%%%%%%%%%%%%%%%%%%%%%%%%%%%%%%%%%%%%%%%%%

  \subsubsection{Methods}

    \vspace{0.5ex}

\hspace{.8\funcindent}\begin{boxedminipage}{\funcwidth}

    \raggedright \textbf{\_\_init\_\_}(\textit{self}, **\textit{params})

    \vspace{-1.5ex}

    \rule{\textwidth}{0.5\fboxrule}
\setlength{\parskip}{2ex}
    Parses a configuration file and any additional keyword parameters to 
    create and initialize a new configuration object.

    The \_\_init\_\_ method accepts a list of optional keyword arguments, 
    reads in additional arguments from a configuration file, and also 
    contains a list of default parameter values. The final value of a 
    particular parameter is set to (in order of precedence):

    \begin{enumerate}

    \setlength{\parskip}{0.5ex}
      \item The supplied keyword parameter, if one is given

      \item The supplied value in the configuration file, if one is given

      \item The built-in default value (if one exists for this parameter)

    \end{enumerate}

    The initialization process does not use the logging system like the 
    rest of Morpher, since the logging system is dependent on configuration
    information supplied here.

    Refer to the documentation for \texttt{ConfigParser} for information on
    how config files are parsed and how key-value pairs can be read and 
    written.

\setlength{\parskip}{1ex}
      \textbf{Parameters}
      \vspace{-1ex}

      \begin{quote}
        \begin{Ventry}{xxxxxxxxxx}

          \item[params]

          Override values for optional keyword arguments

            {\it (type=keyword options)}

          \item[configfile]

          The path to the configuration file

          \item[debug]

          A boolean value enabling debug mode if \textit{True}

          \item[dll]

          The path to the target dll (no default)

          \item[listfile]

          The path to the collection listfile (no default)

        \end{Ventry}

      \end{quote}

      \textbf{Raises}
    \vspace{-1ex}

      \begin{quote}
        \begin{description}

          \item[\texttt{Exception}]

          An exception is raised if a needed parameter is not found in the 
          params, config file, or default values.

        \end{description}

      \end{quote}

      Overrides: ConfigParser.RawConfigParser.\_\_init\_\_

\textbf{Note:} Config defines the default option "basedir" as the path to the current 
working directory. Entries in the config file can use this option to refer 
to other directories relative to the current directory, for example: 
\%(BASEDIR)s{\textbackslash}data



    \end{boxedminipage}

    \label{morpher:misc:config:Config:toString}
    \index{morpher \textit{(package)}!morpher.misc \textit{(package)}!morpher.misc.config \textit{(module)}!morpher.misc.config.Config \textit{(class)}!morpher.misc.config.Config.toString \textit{(method)}}

    \vspace{0.5ex}

\hspace{.8\funcindent}\begin{boxedminipage}{\funcwidth}

    \raggedright \textbf{toString}(\textit{self})

    \vspace{-1.5ex}

    \rule{\textwidth}{0.5\fboxrule}
\setlength{\parskip}{2ex}
    Returns a pretty-printed string suitable for displaying or logging the 
    contents of this Config object

    Returns a string similar to the following:

\begin{alltt}
   Configuration dump: 
   [TEMP]
     basedir : C:{\textbackslash}Users{\textbackslash}Rob{\textbackslash}workspace{\textbackslash}ApiFuzzing
   [directories]
     basedir : C:{\textbackslash}Users{\textbackslash}Rob{\textbackslash}workspace{\textbackslash}ApiFuzzing
     data : C:{\textbackslash}Users{\textbackslash}Rob{\textbackslash}workspace{\textbackslash}ApiFuzzing{\textbackslash}data
     tools : C:{\textbackslash}Users{\textbackslash}Rob{\textbackslash}workspace{\textbackslash}ApiFuzzing ools
     logs : C:{\textbackslash}Users{\textbackslash}Rob{\textbackslash}workspace{\textbackslash}ApiFuzzing{\textbackslash}logs
   [logging]
     basedir : C:{\textbackslash}Users{\textbackslash}Rob{\textbackslash}workspace{\textbackslash}ApiFuzzing
     enabled : yes
     level : debug\end{alltt}

\setlength{\parskip}{1ex}
      \textbf{Return Value}
    \vspace{-1ex}

      \begin{quote}
      Nicely-formatted string containing contents of the Config object

      {\it (type=string)}

      \end{quote}

    \end{boxedminipage}


\large{\textbf{\textit{Inherited from ConfigParser.ConfigParser}}}

\begin{quote}
get(), items()
\end{quote}

\large{\textbf{\textit{Inherited from ConfigParser.RawConfigParser}}}

\begin{quote}
add\_section(), defaults(), getboolean(), getfloat(), getint(), has\_option(), has\_section(), options(), optionxform(), read(), readfp(), remove\_option(), remove\_section(), sections(), set(), write()
\end{quote}

%%%%%%%%%%%%%%%%%%%%%%%%%%%%%%%%%%%%%%%%%%%%%%%%%%%%%%%%%%%%%%%%%%%%%%%%%%%
%%                            Class Variables                            %%
%%%%%%%%%%%%%%%%%%%%%%%%%%%%%%%%%%%%%%%%%%%%%%%%%%%%%%%%%%%%%%%%%%%%%%%%%%%

  \subsubsection{Class Variables}

    \vspace{-1cm}
\hspace{\varindent}\begin{longtable}{|p{\varnamewidth}|p{\vardescrwidth}|l}
\cline{1-2}
\cline{1-2} \centering \textbf{Name} & \centering \textbf{Description}& \\
\cline{1-2}
\endhead\cline{1-2}\multicolumn{3}{r}{\small\textit{continued on next page}}\\\endfoot\cline{1-2}
\endlastfoot\multicolumn{2}{|l|}{\textit{Inherited from ConfigParser.RawConfigParser}}\\
\multicolumn{2}{|p{\varwidth}|}{\raggedright OPTCRE, OPTCRE\_NV, SECTCRE}\\
\cline{1-2}
\end{longtable}

    \index{morpher \textit{(package)}!morpher.misc \textit{(package)}!morpher.misc.config \textit{(module)}!morpher.misc.config.Config \textit{(class)}|)}
    \index{morpher \textit{(package)}!morpher.misc \textit{(package)}!morpher.misc.config \textit{(module)}|)}
